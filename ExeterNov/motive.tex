%%%%%%%%%%%%%%%%%%%%%%%%%%%%%%%%%%%%%%%%%%%%%%%%%%%%%%%%%%%%%%%%%
\begin{slide}
{
    Motivation: Avoid the Pole Problem
    {
        \normalsize\normalfont
        Especially for parallelisation
    }
}
%%%%%%%%%%%%%%%%%%%%%%%%%%%%%%%%%%%%%%%%%%%%%%%%%%%%%%%%%%%%%%%%%

\begin{center}
\includegraphics[width=\linewidth]{plots/meshes+latLon+constant+mesh.pdf}
\end{center}

\end{slide}

%%%%%%%%%%%%%%%%%%%%%%%%%%%%%%%%%%%%%%%%%%%%%%%%%%%%%%%%%%%%%%%%%
\begin{slide}{}
%%%%%%%%%%%%%%%%%%%%%%%%%%%%%%%%%%%%%%%%%%%%%%%%%%%%%%%%%%%%%%%%%

\begin{minipage}[t]{0.24\linewidth}\raggedright

{\Large\bf
Motivation: Resolution where most needed
}

\vspace{0.5in}

NCEP/NCAR precip\\
1 April 2010\\
(mm/day)

\vspace{2.5in}

Mesh of 2,778 cells\\
Equivalent uniform mesh would have 40,962 cells\\

\end{minipage}
%
\begin{minipage}[t]{0.75\linewidth}\raggedright\ \\ \vspace{-0.1in}
\includegraphics[scale=1]{../../../proposals/fellow09/presentation/data/ppt.png}
\includegraphics[scale=1]{../../../proposals/fellow09/presentation/data/mesh.pdf}
\end{minipage}


\end{slide}

%%%%%%%%%%%%%%%%%%%%%%%%%%%%%%%%%%%%%%%%%%%%%%%%%%%%%%%%%%%%%%%%%
\begin{slide}{But fine scale detail is ubiquitous}
%%%%%%%%%%%%%%%%%%%%%%%%%%%%%%%%%%%%%%%%%%%%%%%%%%%%%%%%%%%%%%%%%

\begin{center}
NEODAAS GOES East 075.0W 2 Nov 2008 18:00hours Channel: 1 (Visible)
%Meteosat SEVIRI midday on 2 Nov 2008 Channel: 1 (0.56 - 0.71 $\mu m$ Visible) \\
\includegraphics[scale=0.5]{../../2009/BathMay2009/plots/2008_11_2_1800_GOES12_1_S2.jpg}
\end{center}

\end{slide}

%%%%%%%%%%%%%%%%%%%%%%%%%%%%%%%%%%%%%%%%%%%%%%%%%%%%%%%%%%%%%%%%%%%%%%%
\begin{slide}{Where is improved resolution most needed?}
%%%%%%%%%%%%%%%%%%%%%%%%%%%%%%%%%%%%%%%%%%%%%%%%%%%%%%%%%%%%%%%%%%%%%%%

\vspace{6pt}
{\Large Some small scale features can influence global climate:}
\vspace{6pt}

\begin{itemize}
\item
\begin{minipage}[t]{0.55\linewidth}\Large
Deep atmospheric convection \\(eg supercells are 10s of kms across)
\end{minipage}
\parbox[t]{0.3\linewidth}{\ \\ \vspace{-48pt} \vfill
\includegraphics[width=\linewidth]{../../../proposals/fellow09/presentation/figs/800px-Chaparral_Supercell_2.JPG}
}

\item
\begin{minipage}[t]{0.55\linewidth}\Large
\Large Flow around mountains
\end{minipage}
\parbox[t]{0.3\linewidth}{\ \\ \vspace{-48pt} \vfill
\includegraphics[width=\linewidth]{../../../proposals/fellow09/presentation/figs/20090104_Orographic_clouds__Mt_Cook.JPG}
}

\item
\begin{minipage}[t]{0.55\linewidth}\Large
Temperature inversions with height \\(eg stratocumulus)
\end{minipage}
\parbox[t]{0.3\linewidth}{\ \\ \vspace{-48pt} \vfill
\includegraphics[width=\linewidth]{../../../proposals/fellow09/presentation/figs/stratocumulus.jpg}
}

\end{itemize}

\end{slide}

%%%%%%%%%%%%%%%%%%%%%%%%%%%%%%%%%%%%%%%%%%%%%%%%%%%%%%%%%%%%%%%%%%%%%%%
\begin{slide}{But {\it Deep} Convection is sparse, globally influencial and can be represented with $\sim$km resolution}
%%%%%%%%%%%%%%%%%%%%%%%%%%%%%%%%%%%%%%%%%%%%%%%%%%%%%%%%%%%%%%%%%%%%%%%

\begin{center}
\large
NICAM uses a hexagonal icosahedral mesh at fixed resolutions down to 3.5km on the Earth Simulator and captures features of tropical cloud clusters: \\
(see work of Masaki Satoh et al, JAMSTEC)

\vspace{20pt}
\begin{tabular}{cc}
NICAM at 3.5km & GMS/GOES-9 at Apr. 6, 2004, 00UTC\\
\includegraphics{../../2009/BathMay2009/plots/NICAM.png}
&
\includegraphics{../../2009/BathMay2009/plots/GOES-9.png}
\end{tabular}

\end{center}

\begin{list0}

    \item But these ``Grand Challenge'' experiments are too expensive for operations

    \item And 3.5km is still not really fine enough to {\it resolve} convection

\end{list0}

\end{slide}
